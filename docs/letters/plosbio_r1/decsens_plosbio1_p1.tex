\documentclass[11pt,a4paper]{letter}
\usepackage[top=1.00in, bottom=1.0in, left=1in, right=1in]{geometry}
\usepackage{graphicx}

\signature{Elizabeth M Wolkovich}
%\address{1300 Centre Street \\ Boston, MA, 20131}

\begin{document}
\begin{letter}{}
\includegraphics[width=0.1\textwidth]{/Users/Lizzie/Documents/Professional/images/letterhead/ubc/UBClogo.jpg}\\
\pagenumbering{gobble}
\opening{Dear Dr. Roberts:}
Please consider our revised manuscript, entitled ``A simple explanation for declining temperature sensitivity with warming,'' for consideration in \emph{PLOS Biology}.
\vspace{1.5ex}\\
Climate change has shifted many biological events, and recent observations that species sensitivity to temperature is declining have raised concerns that fundamental biological processes are now also changing. Studies suggest warming has altered the main drivers of leafout in temperate plants and altered carbon uptake in the tundra due to increased light limitation or shifts in photosynthesis.
\vspace{1.5ex}\\
Here we show that these observations---used as evidence of fundamental shifts in underlying biological processes---may be simply the default outcomes of the current methods used to calculate temperature sensitivity with warming. We show theoretically, then through simulations and in empirical data, that observations of declining sensitivities are the null expectation from current methods. This is because many biological events are the outcome of threshold processes and thus will occur more quickly with warming. This in turn will lead to lower estimated responses per degree C when using linear models, without any changes in the underlying biology. 
\vspace{1.5ex}\\
Comments from four reviewers helped us improve our manuscript. We have adjusted our abstract and main text to clarify that our model does not refute that shifting biological processes could underlie observations of declining sensitivities, but provides a simpler explanation for them, and highlights how current methods may make identifying when warming reshapes biological processes especially difficult. We have also added a section on alternative explanations for these declines. In the supplement, we now provide additional simulations that show common hypotheses for declining sensitivities may not produce declining sensitivities and we provide extended methods for our empirical data analyses. Our supplement includes three new or extended figures, extended versions of tables S1-S2 (testing alternative pre-season window lengths) and we provide all code for those interested in using our simulations for their own hypotheses. 
\vspace{1.5ex}\\
We feel the new submission is much improved and detail our changes in the following pages (reviewer comments are in \emph{italics}, while our responses are in regular text). We hope that you will find it suitable for publication in \emph{PLOS Biology} and look forward to hearing from you.
\vspace{1.5ex}\\
Sincerely,\\

\includegraphics[scale=1]{/Users/Lizzie/Documents/Professional/Vitas/Signatures/SignatureLizzieSm.png} \\

Elizabeth M Wolkovich\\
Associate Professor of Forest \& Conservation Sciences\\ 
University of British Columbia
\end{letter}
\end{document}



% \signature{Elizabeth M Wolkovich}
\address{Forest and Conservation Sciences\\
University of British Columbia\\
2424 Main Mall\\
Vancouver, BC V6T 1Z4}
