\documentclass[11pt]{article}
\usepackage[top=1.00in, bottom=1.0in, left=1.1in, right=1.1in]{geometry}
\usepackage{graphicx}
\usepackage{natbib}
\usepackage{amsmath}
\usepackage{lineno}
\usepackage{xr-hyper}
\externaldocument{..//..//decsens}
\newcommand{\R}[1]{\label{#1}\linelabel{#1}}
\newcommand{\lr}[1]{line~\lineref{#1}}
% \usepackage{hyperref}
\setlength\parindent{0pt}

% Need to provide:
% Response to Reviewers 
% In addition to a clean copy of the manuscript, please also upload a 'track-changes' version of your manuscript that specifies the edits made. This should be uploaded as a "Related" file type.  ... https://texblog.org/2018/08/14/track-changes-with-latexdiff/

\begin{document}
Editor and reviewer comments (we provide below the full context of each review) are in \emph{italics}, while our responses are in regular text. \\ % and all in-text citations generally cross-reference to the main text.

{\bf Editor's comments:} \\

\emph{You'll see that while two of the reviewers (\#1 and \#3) are convinced by your study, the other two still need more clarity and detail to persuade them.}\\

We thank the editor for the opportunity to revise our manuscript. We found the reviewers' comments very helpful, with many overlapping requests for clarity or a more careful message. We provide detailed point-by-point responses below.\\

{\bf Reviewer 1 (A Donnelly) comments:} \\

\emph{The authors present a very interesting and important analysis to demonstrate how a decline in temperature sensitivity with warming is obtained using linear regression but the decline disappears when both variables (temperature and leafout dates) are log transformed. This is a very important and timely finding which has implications for future predictions of the impact of climate change on phenology. I would expect this report to motivate researchers to re-examine trends in phenology using this method. The manuscript is extremely well written, explained and presented. Therefore, I recommend this short report for publications without revision.}\\

We thank the reviewer for the positive comments on the manuscript's topic and writing style. We have worked to maintain these good aspects of the manuscript while addressing other reviewer comments. \\

{\bf Reviewer 2 comments:} \\

\emph{This MS provides a simple explanation for the observed non-linearity in biological temperature sensitivity (the observation that at higher temperatures the biological temperature sensitivity per degree Centigrade is decreasing): non-linearity in the biological temperature response. The MS develops a model to prove this point and it is very nicely written. Nevertheless, I found this somewhat uncompelling, because that the explanation appears to be the same as the observation (e.g. crows are black, because crows are black). The main problem appears to be a criticism in how words are used in temporal ecology literature: ``sensitivity'' versus ``response''. Indeed, many authors have used the word ``(in)sensitivity'' rather than the word ``response'' and this could be criticized, because ``(in)sensitivity'' implies some fundamental shift in the temperature response mechanism, even though what we are really seeing is an unchanged response. However, taking this into account the MS is reduced to a semantic discussion, in that not the sensitivity but the response is non-linear.}\\

We understand the reviewer's concerns to be that our finding may seem obvious to some, but we don't see it as a semantics debate for several reasons. First, there is a growing use of shifting or declining sensitivities as evidence of shifts in the underlying biology in the climate change literature \citep[e.g.,][]{fu2015,piao2017}, without a robust interrogation of whether this evidence supports the underlying hypothesis. While we can see that the term `sensitivity' to represent the regression coefficient of a response variable regressed against temperature is often misused, our paper focuses on an issue that remains after the semantics: researchers are using declining magnitudes of response per $^{\circ}$C (declining sensitivities) as evidence of biological shifts, without considering that it may be due to using linear models for non-linear responses. Second, other reviewers of this manuscript here, and additional reviews we gathered from researchers in the field suggest the main crux of our manuscript (that temperature responses are non-linear and thus linear models used to estimate those responses will find declining sensitivities with warming temperatures) was surprising to them. We ourselves were surprised by this result and did not find it intuitive. Third, our simulations show that many models expected to produce declining sensitivities do not (we discuss this more in the following comment), and highlight how critical developing more robust expectations in this area of research may be to progress. Thus, while we understand our findings may not be surprising to some in the field we believe they are novel to many and that highlighting the statistical issue and providing a theoretical underpinning to explain it is an important and useful advance.\\

\emph{Either way, if temperature becomes a non-limiting factor with warming, other environmental factors will inadvertently take over, changing selective pressures on organisms. Therefore, I also do not see how this simple explanation for non-linearity of temperature sensitivity could discredit the conclusion that fundamental aspects of biological processes are being reshaped by climate warming, as the abstract appears to suggest.}\\

We agree with the reviewer that our model does not refute that shifting biological processes could underlie observations of declining sensitivities, and we did not mean to suggest otherwise. We fully agree that with continued warming, we should expect a greater role of factors beyond spring temperatures, but we believe our results highlight that most current methods may not allow us to accurately detect such shifts. To address this we have: (1) adjusted our language in the abstract, which we now close with ``Current methods may thus undermine efforts to identify when and how warming will reshape biological processes'' and in the main text, \lr{R2biomattersstart}-\lr{R2biomattersend}:
\begin{quote}
Our results do not rule out that important shifts in biological processes may underlie observations of declining temperature sensitivities, but provide a simpler explanation for them. Our results highlight how the use of linear models may make identifying when (and potentially why) warming alters underlying biology far more difficult.
\end{quote}
(2) We have provided a longer discussion of the expectation that other environmental factors should play a larger role with warming in the main text (\lr{biomattersstart}-\lr{biomattersend}). (3) We have included an extended discussion and analyses of alternative models in the supplement (see new section `Simulations of common hypotheses for declining sensitivity'). In our previous draft we provided a simulation where reduced overwinter chilling led to a greater thermal sum requirement with warming, we now explain this simulation in more detail. Interestingly, this simulation does not always produce declining sensitivities---for example, we now show in `Effect of increasing thermal sum on sensitivity' that increasing the thermal sum for leafout would produce \emph{increasing} (in magnitude) sensitivities estimated from linear models and the exact predictions for how sensitivities should shift with climate change depend strongly on the underlying model of chilling, forcing and how warming co-varies with each. We now also provide one photoperiod simulation as well, which also highlights the complexity of teasing out other cues.\\

While the focus of our manuscript remains on the statistical issue of using linear models for non-linear temperature responses with warming---which extends beyond spring plant phenology---we hope these examples better highlight the challenges of modeling these processes with current data and show that our assumptions from conceptual models may not always be accurate. We believe this is important because if we are mis-specifying models by assuming linear relationships, we risk missing biological shifts when they are happening.  \\

\emph{The MS could narrow its focus to just being about the difference between ``sensitivity'' and ``response'' to temperature, and that it is important to provide accurate definitions. Moreover, it could urge the field not to overreach in their conclusions on biological temperature responses.}\\

We agree and have worked to balance the manuscript better to contrast our findings with biological expectations given warming, and how well we can detect them. We have also reviewed our uses of the terms, `sensitivity' and `response' for accuracy and clarity; for example, see changes on \lr{semantic1}, \lr{semantic2}, \lr{semantic3} and in the main text figure caption.\\

\emph{On a different note, the number of citations is inadequate. For example, on p4 paragraph 1 is the first mention of "thermal sum" without any citation. This happens more often, and should be revised thoroughly.}\\

Our updated manuscript includes 15 additional citations (our original draft included 12, this draft has 27), see in particular \lr{addcites1}, \lr{addcites2}, \lr{addcites3}. \\% 12 refs originally
%``The 'heat unit approach," in use for over two centuries, is a scheme for studying plant-temperature relationships by the accumulation of daily mean temperatures above a certain threshold temperature during the growing season'' from \citep{wang1960}\\

\emph{I have no small textual comment, as the MS is very well written.}\\

We thank the reviewer for this positive comment and have worked to maintain the writing style, while addressing this and the other reviewers' concerns. \\

{\bf Reviewer 3 (CM Zohner) comments:} \\

\emph{I really like this study. It is an important warning to phenologists that the commonly-used temperature sensitivity metric can be dangerously misleading. The modeling part makes sense to me but I wonder whether more information is required at some parts to allow readers to fully grasp what was done here. For instance, it is not totally clear to me how the PEP data was analysed. How did you determine the preseason lengths (which of course are also problematic per se) on which temperature sensitivities are ultimately based on? I recognize that this is a short report that does not allow for too many details but I think that the Supplementary part should still contain all necessary information to fully understand what has been done here.}\\

We are glad to hear the reviewer appreciated our manuscript and agree we could do more to make our methods clearer. We have updated our supplemental section, `Methods \& results using long-term empirical data (PEP725),' to better explain our methods. We used pre-defined pre-seasons; we are now more explicit about this and for transparency provided two shorter pre-season windows (see updated tables S1-S2). The relevant text now reads:
\begin{quote}
To examine how estimated sensitivities shift over time, we selected sites of two common European tree species (silver birch, \emph{Betula pendula}, and European beech, \emph{Fagus sylvatica}) that have long-term observational data of leafout, through the Pan European Phenology Project \citep[PEP725,][]{Templ2018}. We selected these two species given that they were best represented for consistent data at the same sites over our study years for an early-leafout (\emph{Betula pendula}) and a late-leafout (\emph{Fagus sylvatica}) species (e.g., \emph{Betula pendula} had 17 sites with leafout data from 1950-1960 and 2000-2010, while the next best option for an early-leafout species, \emph{Alnus glutinosa}, had data for only five sites). We used sites with complete leafout data across both our 10-year (and 20-year) windows to avoid possible confounding effects of shifting sites over time (see Tables S1-S2 for numbers of sites per species-window combination). \\

To calculate temperature sensitivities, we used a European-wide gridded climate dataset \emph{\citep[{\normalfont E-OBS},][]{cornes2018}} to extract daily minimum and maximum temperature for the grid cells where observations of leafout for these two species were available (Fig. S5 shows a subset of the climate data for 14 sites used). Determining the appropriate window over which to estimate a temperature sensitivity for spring plant phenology is an area of active research \citep{gusewell2017,xupreseason2018}. Ideally researchers wish to separate windows over which chilling and forcing apply \citep[if they can be cleanly separated,][]{linkosala2008,lundell2020}, but this is generally impossible given our limited understanding of the two processes \citep{chuine2016}. Researchers thus either use a pre-defined spring window \citep[e.g.,][]{zhnag2015,park2018remsens,Park2019,kopp2020} or use a statistical search to determine window attributes; for example, some use a set period then search for a start date \citep[e.g.,][]{Cook:2012pnas,wang2015ecoind}, while others search for both a start date and window length \citep[e.g.,][]{fu2015,tansey2017}. Given the non-identifiability of the simple chilling + forcing model described above (`Simulations of common hypotheses for declining sensitivity'), we do not feel there is sufficient evidence that statistical searches for pre-season windows will select biologically relevant periods related to forcing, and more easily could add additional layers of non-identifiability (to date, most pre-season windows are fit as separate analyses making such issues harder for researchers to identify). Thus, we used pre-determined windows from 1 March to 30 April (60 d; we also present windows of 45 d, from 1 Mar to 15 Apr, and for 31 d, from 15 Mar to 15 Apr, for comparison). Our window represents a period in the spring when forcing is likely the dominant cue, and is similar to many other studies of temperature plants using pre-defined windows \citep[e.g.,][]{prevey2017,Park2019}. We then estimated the sensitivity as a simple linear regression of leafout day of year versus the mean daily temperature over the window (`slope' in Tables S1-S2), or a simple linear regression using logged versions of these predictors (`log-slope' in Tables S1-S2). \\
\end{quote}

\emph{Additionally, I think it is important to emphasize in the text that, although the days per °C metric does not seem to be adequate to detect changes in the phenological responses to temperature, it is still likely that these changes are happening at the moment. You could for example cite some experimental studies, including your own meta-analysis of these studies, that clearly show that the required warming to leaf-out is increasing as winters become shorter, e.g., refs(1-4). The linear dashed line in Figure 3b in ref(5) also demonstrates that, while winter chilling and day length seem to have large effects on the amount of warming (degree-days) to leaf-out, the apparent temperature sensitivity doesn't really change, otherwise one should see a flattening curve. So, while I totally agree that the days per °C metric can be totally misleading, I think the paper would benefit from a short discussion on the importance of spring phenology drivers apart from spring temperature to
emphasize that they are important, although probably not directly visible in the days per °C metric.}\\

This is a good point and somewhat related to points raised by Reviewer \#2. To address it we have: (1) changed the ending the abstract to ``Current methods may thus undermine efforts to identify when and how warming will reshape biological processes,'' (2) provided a longer section in the manuscript, \lr{biomattersstart}-\lr{biomattersend}, which reads (and cites most of the suggested citations):

\begin{quote}
Fundamentally rising temperature should alter many biological processes, making robust methods for identifying these changes critical. In spring plant phenology, where declining sensitivities are often reported \citep{fu2015,piao2017,dai2019ag}, warming may increase the role of `chilling' (determined mainly by winter temperatures) and daylength \citep{Laube:2014a,zohner2016}---potentially increasing the thermal sum required for leafout at lower values of these cues \citep{Polgar2014,zohner2017,flynn2018}. Adjusting our simulations to match this model yielded shifts in sensitivities with warming. Unlike a model with no underlying biological change, however, after correcting for non-linearity, the shifts in sensitivities remained and they occurred in step with the biological change (Fig. S4a, c). In contrast, sensitivities estimated from a linear model showed shifts across the entire range of warming, well before the simulated biological change (Fig. S4a, c). Further, we found that an increase in the thermal sum required for leafout should yield larger in magnitude temperature sensitivities, not smaller, as often expected \citep[e.g.,][]{fu2015}, thus highlighting the complexity of identifying what trends to expect in sensitivities with warming (see `Common hypotheses for declining sensitivity' in Supplementary Information for an extended discussion).
\end{quote}

\emph{References}\\
\emph{1. 	J. Laube, T. H. Sparks, N. Estrella, J. Höfler, D. P. Ankerst, A. Menzel, Chilling outweighs photoperiod in preventing precocious spring development. Glob. Chang. Biol. 20, 170-182 (2014).}\\
\emph{2. 	C. Polgar, A. Gallinat, R. B. Primack, Drivers of leaf-out phenology and their implications for species invasions: Insights from Thoreau's Concord. New Phytol. 202, 106-115 (2014).}\\
\emph{3. 	C. M. Zohner, B. M. Benito, J. D. Fridley, J. C. Svenning, S. S. Renner, Spring predictability explains different leaf-out strategies in the woody floras of North America, Europe and East Asia. Ecol. Lett. 20, 452-460 (2017).}\\
\emph{4. 	D. F. B. Flynn, E. M. Wolkovich, Temperature and photoperiod drive spring phenology across all species in a temperate forest community. New Phytol. 219, 1353-1362 (2018).}\\
\emph{5. 	C. M. Zohner, L. Mo, T. A. M. Pugh, J. F. Bastin, T. W. Crowther, Interactive climate factors restrict future increases in spring productivity of temperate and boreal trees. Glob. Chang. Biol. 26, 4042-4055 (2020).}\\

These references are now all cited in the manuscript.\\

{\bf Reviewer 4 comments:} \\

\emph{In this short manuscript, the authors intended to show that rising temperatures combined with linear estimates based on calendar time are expected to produce observations of declining sensitivity without any shift in the underlying biology. While the topic is very interesting and the conclusion is eye-catching, I found that the reasoning and presentation of the manuscript deserved much more reconsideration and the conclusions were not convincing.}\\

\emph{My general feeling is that the manuscript was presented in a form that was very hard to follow. For example, the authors presented a stochastic experiment, whose details were buried in the supplementary information. Many critical details of the experiments were missing, making it difficult to assess. The authors later argued a ``correction'' for non-linearity, but the theorem of this correction and its caveats were not presented. Coming after that, the authors intended to test their theory with the observation dataset (PEP725), but critical details on how they calculated the temperature sensitivity was not clear, which, as Keenan et al. (2020) mention, is a fundamental issue in deriving temperature sensitivity. I have to read the short manuscript for several times, tracing backward and forward between the main-text and SI, and taking some guesses to what they have done. Thus, I strongly recommend the authors to make a full-length paper with clear presentation on their methods
and reasoning. In its current form, the manuscript is not publishable in my opinion.}\\

We appreciate the reviewer's concerns that we should have provided clearer methods for our PEP725 analyses, and this was also a concern of Reviewer \#3. To address this we have greatly extended the section in the supplement on the PEP725 analysis, providing full methods (see also the response above to Reviewer 3's first point, where we show much of the relevant text). \\

Our stochastic experiment is designed to be simple---a model where an event is triggered based on a thermal sum threshold, as we describe in the main text. We fully describe the stochastic model both mathematically and in a longer text format in the supplement. The correction we suggest (log transformation) is the natural transformation for an inverse relationship (by definition a log linearizes this relationship) so we are unsure how to further describe this. While we could move the mathematical description of the model into the main text, we feel that will narrow the readership unnecessarily. We provide the supplement math as a more developed proof for those interested. Based on several reviews by those in this research area, including those of the other three reviewers here, we believe this format of the manuscript allows readers to understand our main point, analyses and findings, though we are happy to defer to the editor on the best format for the readership and make changes as requested.\\

\emph{A major argument made by the authors was a stochastic experiment: when leafout date was determined by a constant cumulative daily temperature, a declining apparent temperature sensitivity could be derived. However, this argument has two limitations: (1) the assumption of constant threshold of cumulative heat is a very strong one, and not supported by observations (e.g. Fu et al., 2014); (2) even we accepted the strong assumption, it does not prove that the observed decline in apparent temperature sensitivity comes from unchanged heat accumulation. As the authors showed, increasing heat requirements could indeed yield declining apparent temperature sensitivity. As a reader, I do not see the value of this stochastic experiment, which did not successfully prove the points made by the author. }\\

We did not mean to suggest that declining sensitivities are necessarily driven by the model we present, but rather provide a simpler explanation for them, and we should have made this clearer. As noted above in our reply to Reviewer 2, to address this we have: (1) adjusted our language in the abstract, which we now close with ``Current methods may thus undermine efforts to identify when and how warming will reshape biological processes'' and in the main text, \lr{R2biomattersstart}-\lr{R2biomattersend}:
\begin{quote}
Our results do not rule out that important shifts in biological processes may underlie observations of declining temperature sensitivities, but provide a simpler explanation for them. Importantly, our results highlight how the use of linear models may make identifying when (and potentially why) warming alters underlying biology far more difficult.
\end{quote}
Regarding the assumption of the constant threshold of cumulative heat, we have added a new section (`Effect of increasing thermal sum on sensitivity') that examines the effects of relaxing this assumption and find that, perhaps contrary to expectations, a shifting thermal sum would produce increasing, not declining, sensitivities. %We did not include this before as we aimed to test a simple model so that we could understand expected shifts in sensitivities without this effect. 
We are not aware of strong observational evidence that this threshold is varying (Fu \emph{et al.} 2014 in \emph{PLOS One} and Fu \emph{et al.} 2015 in \emph{Nature} both use methods similar to the one we critique here to suggests shifting thresholds, but we may be reviewing the wrong papers), though we now clearly state and cite experimental evidence that these thresholds can shift (\lr{biomattersstart}-\lr{R4end}):
\begin{quote}
Fundamentally rising temperature should alter many biological processes, making robust methods for identifying these changes critical. In spring plant phenology, where declining sensitivities are often reported \citep{fu2015,piao2017,dai2019ag}, warming may increase the role of `chilling' (determined mainly by winter temperatures) and daylength \citep{Laube:2014a,zohner2016}---potentially increasing the thermal sum required for leafout at lower values of these cues \citep{Polgar2014,zohner2017,flynn2018}.
\end{quote} 
In addressing concerns from Reviewers 2-3, we also provide an extended discussion of the complexity of outcomes from models that predict shifting thermal sums based on other cues for spring phenological events. While many of the authors ourselves assumed that sensitivities to spring temperatures should decline as other cues come into play, this is not always the case given the co-variation of many of these cues, and highlights the value of explicit models to challenge assumptions. \\

\emph{In the next paragraph, the authors argued that a log transformation could make the declining temperature sensitivity disappear. While this may be true, an issue may arise as the log transform would make the regression sensitive to a few outliers and it will necessarily hide change in temperature sensitivity, if it happens. So I hardly see how the disappearance of linearity after log-transformation proves that non-linearity did not exist in the observations.}\\

Log transformations make regressions less (not more) sensitive to outliers, which we assume is what the reviewer means, though we're not clear if the concern here is regarding outliers in the predictor or response, or both. Our argument is that a log is the correct transformation for the data given the model that plants leafout after reaching a threshold thermal sum. As mentioned above, we do not mean to suggest that our model in any way proves that there are not biological shifts underlying the PEP 725 data, but that our current statistical methods may not be robust to detecting these changes, and we have edited the abstract and main text to clarify this. \\

\emph{Also about the analyses on the observation dataset: the PEP dataset includes much more species than the two used here, why selectively analyzing these two species?}\\

Good catch, we now clarify our methods regarding species selection in the supplement (`Methods \& results using long-term empirical data (PEP725)'). We write:
\begin{quote}
 We selected these two species given that they were best represented for consistent data at the same sites over our study years for an early-leafout (\emph{Betula pendula}) and a late-leafout (\emph{Fagus sylvatica}) species (e.g., \emph{Betula pendula} had 17 sites with leafout data from 1950-1960 and 2000-2010, while the next best option for an early-leafout species, \emph{Alnus glutinosa}, had data for only five sites). 
\end{quote}
For clarity we provide on the following pages the number of sites for other major species in the PEP725 data, including our analyses of the species with the third most data for our analyses, \emph{Quercus robur}. Additionally, we now provide all related code in a GitHub repository, referenced in the main text and supplement.\\

\emph{Overall, the stochastic experiment showed that declining apparent temperature sensitivity could have some alternative explanation, but it is far from sufficient to conclude that the observed declining temperature sensitivity was from the authors' assumption. The empirical analyses were not performed in a transparent and convincing way. Therefore, I could not draw similar conclusions as the authors did.}\\

Our goal with the stochastic experiment was to show, as this reviewer writes, ``that declining apparent temperature sensitivity could have some alternative explanation,'' and thus we believe our manuscript did communicate our main finding. This observation is important because it highlights that commonly used statistical approaches are unable to distinguish between non-linear trends and underlying shifts in biological process driving phenology. We have worked to adjust our language to make it clear that we do not mean to state that we have shown this is the definitive explanation, but instead that it is an explanation that should be better considered in analyses using declining sensitivities with warming as evidence of shifting biology with warming. We believe our revisions have further clarified our methods, analyses and conclusions and thank the reviewer for this.

\bibliography{..//..//..//refs/decsens.bib}
\bibliographystyle{..//..//..//refs/bibstyles/amnat.bst}

\end{document}
