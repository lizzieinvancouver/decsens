\documentclass[11pt,a4paper]{article}
\usepackage[top=1.00in, bottom=1.0in, left=1in, right=1in]{geometry}
\usepackage{graphicx}
\usepackage{sectsty,setspace,natbib,wasysym} 

\begin{document}
\bibliographystyle{/Users/Lizzie/Documents/EndnoteRelated/Bibtex/styles/naturemag}
\begin{figure}[htbp]
\hspace*{14cm}                                                           
\includegraphics[width=0.1\textwidth]{/Users/Lizzie/Documents/Professional/images/letterhead/ubc/UBClogo.jpg}
\end{figure}
\vspace{-10ex}
\begin{small}
\noindent 2424 Main Mall \\
\noindent Vancouver, BC Canada V6T 1Z4\\
\noindent Ph: 604.827.5246\\
\end{small}
\vspace{2ex}\\
\pagenumbering{gobble}
\noindent Dear Dr. Pariente:
\vspace{1.5ex}\\
Please consider our manuscript, entitled ``A simple explanation for declining temperature sensitivity with warming,'' for consideration as an initial submission for \emph{PLOS Biology}. % 
\vspace{1.5ex}\\
Climate change has shifted many biological events \citep{IPCC:2014sm}, and recent observations that species sensitivity to temperature is declining have raised concerns that fundamental biological processes are now also changing. Studies suggest warming has altered the main drivers of leafout in temperate plants \citep{fu2015,gusewell2017,Samplonius:2018aa,vitasse2018} and altered carbon uptake in the tundra due to increased light limitation or shifts in photosynthesis \citep{piao2017,Zhu2019}. The potential for such ecosystem effects is alarming, but they are rarely reported with strong evidence beyond declines in temperature sensitivity.
\vspace{1.5ex}\\
Here we show that these observations---used as evidence of fundamental shifts in underlying biological processes---are an illusion. They are instead the default outcomes of the current methods used to calculate temperature sensitivity with warming. We show theoretically, then through simulations and in empirical data \citep[using the same dataset used in][]{fu2015}, that observations of declining sensitivities are the null expectation from current methods. This is because many biological events are the outcome of threshold processes and thus will occur more quickly with warming. This in turn will lead to lower estimated responses per degree C when using linear models, without any changes in the underlying biology. When using simple non-linear methods, we find no evidence of declining sensitivities in empirical data. % Effectively, warming acts to step on the biological accelerator, making the use of classic calendar time problematic. 
\vspace{1.5ex}\\
We believe this confusion in interpreting observed declining sensitivities is currently rampant throughout the phenological \citep[e.g.,][]{fu2015,Samplonius:2018aa,vitasse2018,meng2020} and related literature \citep[e.g.,][]{piao2017} and may apply more broadly to many fields using linear methods and fixed time periods with warming. Our understanding of this problem only came through close cross-disciplinary collaboration between biologists (Wolkovich group) and statisticians (J. Auerbach and A. Gelman). 
\vspace{1.5ex}\\
%We have suggested several potential reviewers through the online submission site (Julio Betancourt, Theresa Crimmins, Alison Donnelly,  Jarrod Hadfield). 
Prior to submission Jonathan Davies, David Lipson and Christy Rollinson reviewed the manuscript. This manuscript is not under consideration elsewhere. All authors approved of this version for submission. Empirical data are already publicly available through the PEP 725 portal, and we provide a link in the main text to our simulation code. The top three references in this area are Fu \emph{et al.} 2015 (\emph{Nature}), Piao \emph{et al.} 2017 (\emph{Nature Climate Change}) and Meng \emph{et al.} 2020 (\emph{PNAS}).\\
\vspace{1.5ex}\\
Sincerely,\\

\includegraphics[scale=1]{/Users/Lizzie/Documents/Professional/Vitas/Signatures/SignatureLizzieSm.png} \\

\noindent Elizabeth M Wolkovich\\
Associate Professor of Forest \& Conservation Sciences\\ 

\bibliography{..//..//refs/decsens.bib}
\end{document}



% \signature{Elizabeth M Wolkovich}
\address{Forest and Conservation Sciences\\
University of British Columbia\\
2424 Main Mall\\
Vancouver, BC V6T 1Z4}
