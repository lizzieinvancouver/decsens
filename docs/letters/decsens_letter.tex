\documentclass[11pt,a4paper]{article}
\usepackage[top=1.00in, bottom=1.0in, left=1in, right=1in]{geometry}
\usepackage{graphicx}
\usepackage{sectsty,setspace,natbib,wasysym} 

\begin{document}
\bibliographystyle{/Users/Lizzie/Documents/EndnoteRelated/Bibtex/styles/naturemag}
\begin{figure}[htbp]
\hspace*{14cm}                                                           
\includegraphics[width=0.1\textwidth]{/Users/Lizzie/Documents/Professional/images/letterhead/ubc/UBClogo.jpg}
\end{figure}
\vspace{-10ex}
\begin{small}
\noindent 2424 Main Mall \\
\noindent Vancouver, BC Canada V6T 1Z4\\
\noindent Ph: 604.827.5246\\
\end{small}
\vspace{2ex}\\
\pagenumbering{gobble}
\noindent Dear Dr. Sudgen:
\vspace{1.5ex}\\
Please consider our manuscript, entitled ``A simple explanation for declining temperature sensitivity with warming,'' for publication as a Report in \emph{Science}. 
\vspace{1.5ex}\\
Climate change has shifted many biological events \citep{IPCC:2014sm}, and recent observations that species sensitivity to temperature is declining has raised concerns that fundamental biological processes are now also changing. Studies suggest warming has altered the main drivers of leafout in temperate plants \citep{fu2015,gusewell2017,Samplonius:2018aa,vitasse2018} and altered carbon uptake in the tundra possibly due to increased light limitation or shifts in photosynthesis \citep{piao2017,Zhu2019}. The potential for such ecosystem effects is alarming, but they are rarely reported with strong evidence beyond declines in temperature sensitivity.
\vspace{1.5ex}\\
Here we provide a simpler explanation for declining temperature sensitivity with warming: the use of linear models to model non-linear temperature responses. We show theoretically, then through simulations and in empirical data \citep[using the same dataset used in][]{fu2015}, that observations of declining sensitivities are the null expectation from current methods. This is because many biological events are the outcome of threshold processes and thus will occur more quickly with warming. This in turn will lead to statistically lower estimated responses per degree C when using linear models, without any changes in the underlying biology. Effectively, warming acts to step on the biological accelerator, making the use of classic calendar time combined with linear models problematic. 
\vspace{1.5ex}\\
We believe this confusion on the interpretation of observed declining sensitivities is currently rampant throughout the phenological \citep[e.g.,][]{fu2015,Samplonius:2018aa,vitasse2018,meng2020} and related literature \citep[e.g.,][]{piao2017} and may apply more broadly to many fields using linear methods and fixed time periods with warming. Our understanding this problem only came through close cross-disciplinary collaboration between biologists (Wolkovich group) and statisticians (J. Auerbach and A. Gelman). 
\vspace{1.5ex}\\
We have suggested three potential reviewers through the online submission site (Theresa Crimmins, Alison Donnelly, Mark Schwartz). Prior to submission Jonathan Davies, David Lipson and Christy Rollinson reviewed the manuscript. This manuscript is not under consideration elsewhere, nor has it been previously submitted. Empirical data are already publicly available through the PEP 725 portal, and we provide a link in the main text to our simulation code. All authors approved of this version for submission. \\
\vspace{1.5ex}\\
Sincerely,\\

\includegraphics[scale=1]{/Users/Lizzie/Documents/Professional/Vitas/Signatures/SignatureLizzieSm.png} \\

\noindent Elizabeth M Wolkovich\\
Associate Professor of Forest \& Conservation Sciences\\ 

\bibliography{..//..//refs/decsens.bib}
\end{document}



% \signature{Elizabeth M Wolkovich}
\address{Forest and Conservation Sciences\\
University of British Columbia\\
2424 Main Mall\\
Vancouver, BC V6T 1Z4}
