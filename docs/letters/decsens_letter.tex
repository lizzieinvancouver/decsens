\documentclass[11pt,a4paper]{article}
\usepackage[top=1.00in, bottom=1.0in, left=1in, right=1in]{geometry}
\usepackage{graphicx}
\usepackage{sectsty,setspace,natbib,wasysym} 

\begin{document}
\bibliographystyle{/Users/Lizzie/Documents/EndnoteRelated/Bibtex/styles/science}
\begin{figure}[htbp]
\hspace*{14cm}                                                           
\includegraphics[width=0.1\textwidth]{/Users/Lizzie/Documents/Professional/images/letterhead/ubc/UBClogo.jpg}
\end{figure}
\vspace{-10ex}
\begin{small}
\noindent 2424 Main Mall \\
\noindent Vancouver, BC Canada V6T 1Z4\\
\noindent Ph: 604.827.5246\\
\end{small}
\vspace{2ex}\\
\pagenumbering{gobble}
\noindent Dear Dr. Sudgen:
\vspace{1.5ex}\\
% Ref decsens.bib and these ...
% \cite{fu2015}
% \cite{meng2020}
% Problem now extending to evidence of weaking carbon uptake reports \cite{piao2017,Zhu2019}
% Mention maybe smaller than a report? Other formats?
Warming acts to step on the biological accelerator, and makes the use of classic calendar time precarious.

Please consider our manuscript, entitled ``The illusion of declining temperature sensitivity with warming,'' for publication as a Report (?) in \emph{Science}. 
\vspace{1.5ex}\\
The biological consequences of climate change are profoundly
important, with studies based on both experimental
\citep{Harte:1995lj,Knapp:2002py,Harmon:2009pd,Langley:2010cr} and
observational
\citep{Fitter:2002ux,Loarie:2009ax,Lewis:2009lq,Bond-Lamberty:2010dq,Boyce:2010rr}
approaches often appearing in high profile journals. We present the first
large-scale effort---using newly compiled data covering over 200
years and comprising 1,560 species---to compare
biotic responses in warming experiments to observational trends. We
show that experimental studies underpredict advances in leaf and flowering
phenology with global warming by at
least \(4.6X\) compared to long-term studies. This suggests that the magnitude of plant
responses to climate change cannot be explained or predicted by
experimental warming studies. Further, because phenology is the most frequently reported measure of
plant responses to climate,
across both experiments and observations, it is our 
best option to robustly examine differences between study types. We
fully expect that this discrepancy in phenological responses is
mirrored by 
other critical metrics, such as productivity, for which such extensive data
are not available. This study is a key product of an interdisciplinary NCEAS working group (`Forecasting Phenology') and includes perspectives from some of the top ecologists, climate scientists and evolutionary biologists working on phenology and climate change.
\vspace{1.5ex}\\
\noindent Our main result has
extensive consequences for predictions of species diversity, ecosystem
services and global models of future change, with clear implications
for how climate change experiments and long-term monitoring are
designed and interpreted. As such, we expect our manuscript to be
controversial and challenging to current practitioners in the
field. We believe, however, that the accuracy of warming experiments
when compared to long-term observational data is an important issue as
there is much at stake in predicting the ecological outcomes of global
change.  Given this, we especially request that reviewers
represent both those experienced in observational as well as
experimental phenological studies who can provide a fair critique of
what we have already found to be quite controversial results among
more experimentally-focused ecologists.  We have suggested a mix (David Inouye, Jennifer Dunne, F. Stuart Chapin, This Rutishauser, Eric Post) of potential reviewers through the online submission site. Prior to submission Matthew Ayres, Lara Kueppers, David Moore, and Mary O'Connor reviewed the manuscript.\\
\vspace{1.5ex}\\
\noindent This manuscript is not under consideration elsewhere, nor has it been previously submitted. All authors approved of this version for submission. All data underlying the analyses presented here will be publicly available via the Knowledge Network for Biodiversity (KNB, http://knb.ecoinformatics.org/) within 6 months.\\
\vspace{1.5ex}\\
Sincerely,\\

\includegraphics[scale=1]{/Users/Lizzie/Documents/Professional/Vitas/Signatures/SignatureLizzieSm.png} \\

Elizabeth M Wolkovich\\
Associate Professor of Forest \& Conservation Sciences\\ 
University of British Columbia\\

\begin{footnotesize}
\bibliography{/Users/Lizzie/Documents/EndnoteRelated/Bibtex/LizzieMainMinimal}
\end{footnotesize}
\end{document}



% \signature{Elizabeth M Wolkovich}
\address{Forest and Conservation Sciences\\
University of British Columbia\\
2424 Main Mall\\
Vancouver, BC V6T 1Z4}
